%%%%%%%%%%%%%%%%%%%%%%%%%%%%%%%%%%%%%%%%%%%%%%%%%%%%%%%%%%%%%%%%%%%%%%%%%%%%%
%% Original default rstudio/pandoc latex file
%% upated by @jhollist 09/15/2014
%% inspired by @cboetting https://github.com/cboettig/template and
%% @rmflight blog posts:
%% http://rmflight.github.io/posts/2014/07/analyses_as_packages.html 
%% http://rmflight.github.io/posts/2014/07/vignetteAnalysis.html).  
%%%%%%%%%%%%%%%%%%%%%%%%%%%%%%%%%%%%%%%%%%%%%%%%%%%%%%%%%%%%%%%%%%%%%%%%%%%%%

\documentclass[11pt,a4paper]{article}
\usepackage[T1]{fontenc}
\usepackage{lmodern}
\usepackage{amssymb,amsmath}
\usepackage{ifxetex,ifluatex}
\usepackage{fixltx2e} % provides \textsubscript
% use upquote if available, for straight quotes in verbatim environments
\IfFileExists{upquote.sty}{\usepackage{upquote}}{}
\ifnum 0\ifxetex 1\fi\ifluatex 1\fi=0 % if pdftex
  \usepackage[utf8]{inputenc}
\else % if luatex or xelatex
  \ifxetex
    \usepackage{mathspec}
    \usepackage{xltxtra,xunicode}
  \else
    \usepackage{fontspec}
  \fi
  \defaultfontfeatures{Mapping=tex-text,Scale=MatchLowercase}
  \newcommand{\euro}{€}
\fi
% use microtype if available
\IfFileExists{microtype.sty}{\usepackage{microtype}}{}
\usepackage[margin=1in]{geometry}
\usepackage{longtable,booktabs}
\usepackage{graphicx}
% Redefine \includegraphics so that, unless explicit options are
% given, the image width will not exceed the width of the page.
% Images get their normal width if they fit onto the page, but
% are scaled down if they would overflow the margins.
\makeatletter
\def\ScaleIfNeeded{%
  \ifdim\Gin@nat@width>\linewidth
    \linewidth
  \else
    \Gin@nat@width
  \fi
}
\makeatother
\let\Oldincludegraphics\includegraphics
{%
 \catcode`\@=11\relax%
 \gdef\includegraphics{\@ifnextchar[{\Oldincludegraphics}{\Oldincludegraphics[width=\ScaleIfNeeded]}}%
}%
\ifxetex
  \usepackage[setpagesize=false, % page size defined by xetex
              unicode=false, % unicode breaks when used with xetex
              xetex]{hyperref}
\else
  \usepackage[unicode=true]{hyperref}
\fi
\hypersetup{breaklinks=true,
            bookmarks=true,
            pdfauthor={},
            pdftitle={A template for writing manuscripts in Rmarkdown},
            colorlinks=true,
            citecolor=blue,
            urlcolor=blue,
            linkcolor=magenta,
            pdfborder={0 0 0}}
\urlstyle{same}  % don't use monospace font for urls
\setlength{\parindent}{0pt}
\setlength{\parskip}{6pt plus 2pt minus 1pt}
\setlength{\emergencystretch}{3em}  % prevent overfull lines
\setcounter{secnumdepth}{0}

%%%%%%%%%%%%%%%%%%%%%%%%%%%%%%%%%%%%%%%%%%%%%%%%%%%%%%%%
%Changes borrowed from @cboettig, added by @jhollist 
% A modified page layout 
\textwidth 6.75in
\oddsidemargin -0.15in
\evensidemargin -0.15in
\textheight 9in
\topmargin -0.5in
\usepackage{lineno} % add 
  \linenumbers % turns line numbering on 
%%%%%%%%%%%%%%%%%%%%%%%%%%%%%%%%%%%%%%%%%%%%%%%%%%%%%%%%

%%%%%%%%%%%%%%%%%%%%%%%%%%%%%%%%%%%%%%%%%%%%%%%%%%%%%%%%
%%Packages and layout changes by @jhollist 09/15/2014
\usepackage{ragged2e}
\usepackage[font=normalsize]{caption}
  \usepackage[doublespacing]{setspace}
\usepackage{parskip}
\usepackage{fancyhdr}
\pagestyle{fancy}
\fancyhf{}
\renewcommand{\headrulewidth}{0pt}
  \rfoot{\today}
\lfoot{\thepage}
%%Changed default abstract width and added lines
\renewenvironment{abstract}{
  \hfill\begin{minipage}{1\textwidth}
  \rule{\textwidth}{1pt}\vspace{5pt}
  \normalsize
  \begin{justify}
  \bfseries\abstractname\vspace{5pt}
  \end{justify}}
  {\par\noindent\rule{\textwidth}{1pt}\end{minipage}
}
%%%%%%%%%%%%%%%%%%%%%%%%%%%%%%%%%%%%%%%%%%%%%%%%%%%%%%%%

\title{A template for writing manuscripts in Rmarkdown}
\author{
Jeffrey W. Hollister
Francisco Rodriguez-Sanchez
}
\date{}
% Allowing for landscape pages
\usepackage{lscape}
\newcommand{\blandscape}{\begin{landscape}}
\newcommand{\elandscape}{\end{landscape}}

% Left justification of the text: see https://www.sharelatex.com/learn/Text_alignment
% \usepackage[document]{ragged2e} % already in the latex template
\newcommand{\bleft}{\begin{flushleft}}
\newcommand{\eleft}{\end{flushleft}}

\begin{document}
%%Edited by @jhollist 09/15/2014
%%Adds title from YAML
\begin{singlespace}
\begin{center}
\huge A template for writing manuscripts in Rmarkdown
\end{center}
%%Adds Author, correspond email asterisk, and affilnum from YAML
\begin{center}
\large
Jeffrey W. Hollister \textsuperscript{1,2} 
Francisco Rodriguez-Sanchez \textsuperscript{*} \textsuperscript{3} 
\end{center}
%%Adds affiliations from YAML
\begin{justify}
\footnotesize \emph{ 
\\*
\textsuperscript{1}US Environmental Protection Agency, Office of Research and Development,
National Health and Environmental Effects Research Laboratory, Atlantic
Ecology Division, 27 Tarzwell Drive Narragansett, RI, 02882, USA\\*
\\*
\textsuperscript{2}Big Name University, Department of R, City, BN, 01020, USA\\*
\\*
\textsuperscript{3}Estacion Biologica de Donana (EBD-CSIC), E-41092 Sevilla, Spain\\*
}
%%Adds corresponding author email(s) from YAML
\newcounter{num}
\setcounter{num}{1}
\\[0.1cm]
\footnotesize \emph{ 
\ifnum\value{num}=1%
\textsuperscript{*} corresponding author:
\fi
\href{mailto:f.rodriguez.sanc@gmail.com}{\nolinkurl{f.rodriguez.sanc@gmail.com}}
\stepcounter{num}
}
\end{justify}
%%Adds date from YAML
\normalsize

\end{singlespace}


\singlespace

\vspace{2mm}

\hrule

Write your abstract here.

\vspace{3mm}

\hrule

\emph{Keywords}: rmarkdown, reproducible science

\doublespace

\bleft

\section{INTRODUCTION}\label{introduction}

Write your introduction here. You can cite bibliography like this (Yan
and Gerstein 2011, Sutherland et al. 2011), if you provide a
\texttt{BibTeX} file with references. See
\url{http://rmarkdown.rstudio.com/authoring_bibliographies_and_citations.html}
for more information. Or you could also use
\href{https://cran.r-project.org/web/packages/knitcitations/index.html}{knitcitations}
or
\href{https://cran.r-project.org/web/packages/RefManageR/index.html}{RefManageR}
to fetch bibliographic metadata automatically from the web. For example,
citing a paper can be as easy as providing its DOI (Clark and Gelfand
2006) or even just a few keywords (Ricklefs 2008). They will then
automagically appear in the list of cited references.

You can even specifiy the desired output format for your bibliography by
including a style file for a specific journal (e.g. ``ecology.csl'').
Many different bibliography styles (CSL files) can be obtained at
\url{http://citationstyles.org/} or
\url{https://github.com/citation-style-language/styles}.

\section{METHODS}\label{methods}

\subsection{Study Area}\label{study-area}

We worked in a \textbf{beautiful} place with lots of trees, like
\emph{Quercus suber} and \emph{Laurus nobilis}.

\subsection{Data collection and
analysis}\label{data-collection-and-analysis}

We applied a linear model where

\[
y_{i} = \alpha + \beta*x_{i} 
\]

We used the statistical language \texttt{R} (R Core Team 2016) for all
our analyses. These were implemented in dynamic rmarkdown documents
using \texttt{knitr} (Xie 2014, 2015, 2016) and \texttt{rmarkdown}
(Allaire et al. 2016) packages. All the multilevel models were fitted
with \texttt{lme4} (Bates et al. 2015).

\section{RESULTS}\label{results}

Trees in forest A grew taller than those in forest B (mean height: 25
versus 13 m). And many more cool results that get updated dynamically.

\section{DISCUSSION}\label{discussion}

Discuss.

\section{CONCLUSIONS}\label{conclusions}

\section{ACKNOWLEDGEMENTS}\label{acknowledgements}

\section{REFERENCES}\label{references}

\hypertarget{refs}{}
\hypertarget{ref-Allaire_2016}{}
Allaire, J., J. Cheng, Y. Xie, J. McPherson, W. Chang, J. Allen, H.
Wickham, A. Atkins, and R. Hyndman. 2016. Rmarkdown: Dynamic documents
for r.

\hypertarget{ref-Bates_2015}{}
Bates, D., M. Mächler, B. Bolker, and S. Walker. 2015. Fitting linear
mixed-effects models using lme4. Journal of Statistical Software
67:1--48.

\hypertarget{ref-Clark_2006}{}
Clark, J. S., and A. E. Gelfand. 2006. A future for models and data in
environmental science. Trends in Ecology \& Evolution 21:375--380.

\hypertarget{ref-R_Core_Team_2016}{}
R Core Team. 2016. R: A language and environment for statistical
computing. R Foundation for Statistical Computing, Vienna, Austria.

\hypertarget{ref-ricklefs2008disintegration}{}
Ricklefs, R. 2008. Disintegration of the ecological community: American
society of naturalists sewall wright award winner address. The American
Naturalist 172:741--750.

\hypertarget{ref-Sutherland2011}{}
Sutherland, W. J., D. Goulson, S. G. Potts, and L. V. Dicks. 2011.
Quantifying the impact and relevance of scientific research. PLoS ONE
6:e27537.

\hypertarget{ref-Xie_2014}{}
Xie, Y. 2014. Knitr: A comprehensive tool for reproducible research in
R. \emph{in} V. Stodden, F. Leisch, and R. D. Peng, editors.
Implementing reproducible computational research. Chapman; Hall/CRC.

\hypertarget{ref-Xie_2015}{}
Xie, Y. 2015. Dynamic documents with R and knitr. 2nd editions. Chapman;
Hall/CRC, Boca Raton, Florida.

\hypertarget{ref-Xie_2016}{}
Xie, Y. 2016. Knitr: A general-purpose package for dynamic report
generation in r.

\hypertarget{ref-Yan2011}{}
Yan, K.-K., and M. Gerstein. 2011. The spread of scientific information:
Insights from the web usage statistics in plos article-level metrics.
PLoS ONE 6:e19917.

\eleft

\clearpage

\listoftables

\newpage

\begin{longtable}[]{@{}rrrrl@{}}
\caption{A glimpse of the famous \emph{Iris} dataset.}\tabularnewline
\toprule
Sepal.Length & Sepal.Width & Petal.Length & Petal.Width &
Species\tabularnewline
\midrule
\endfirsthead
\toprule
Sepal.Length & Sepal.Width & Petal.Length & Petal.Width &
Species\tabularnewline
\midrule
\endhead
5.1 & 3.5 & 1.4 & 0.2 & setosa\tabularnewline
4.9 & 3.0 & 1.4 & 0.2 & setosa\tabularnewline
4.7 & 3.2 & 1.3 & 0.2 & setosa\tabularnewline
4.6 & 3.1 & 1.5 & 0.2 & setosa\tabularnewline
5.0 & 3.6 & 1.4 & 0.2 & setosa\tabularnewline
5.4 & 3.9 & 1.7 & 0.4 & setosa\tabularnewline
\bottomrule
\end{longtable}

\newpage

\begin{longtable}[]{@{}lrrrrrrrrrrr@{}}
\caption{Now a subset of mtcars dataset.}\tabularnewline
\toprule
& mpg & cyl & disp & hp & drat & wt & qsec & vs & am & gear &
carb\tabularnewline
\midrule
\endfirsthead
\toprule
& mpg & cyl & disp & hp & drat & wt & qsec & vs & am & gear &
carb\tabularnewline
\midrule
\endhead
Merc 280 & 19.2 & 6 & 167.6 & 123 & 3.92 & 3.440 & 18.30 & 1 & 0 & 4 &
4\tabularnewline
Merc 280C & 17.8 & 6 & 167.6 & 123 & 3.92 & 3.440 & 18.90 & 1 & 0 & 4 &
4\tabularnewline
Merc 450SE & 16.4 & 8 & 275.8 & 180 & 3.07 & 4.070 & 17.40 & 0 & 0 & 3 &
3\tabularnewline
Merc 450SL & 17.3 & 8 & 275.8 & 180 & 3.07 & 3.730 & 17.60 & 0 & 0 & 3 &
3\tabularnewline
Merc 450SLC & 15.2 & 8 & 275.8 & 180 & 3.07 & 3.780 & 18.00 & 0 & 0 & 3
& 3\tabularnewline
Cadillac Fleetwood & 10.4 & 8 & 472.0 & 205 & 2.93 & 5.250 & 17.98 & 0 &
0 & 3 & 4\tabularnewline
Lincoln Continental & 10.4 & 8 & 460.0 & 215 & 3.00 & 5.424 & 17.82 & 0
& 0 & 3 & 4\tabularnewline
\bottomrule
\end{longtable}

\clearpage

\listoffigures

\newpage

\begin{figure}[htbp]
\centering
\includegraphics{output/figures/Fig1-1.pdf}
\caption{Just my first figure with a very fantastic caption.}
\end{figure}

\newpage

\blandscape

\begin{figure}[htbp]
\centering
\includegraphics{output/figures/Fig2-1.pdf}
\caption{Second figure in landscape format.}
\end{figure}

\elandscape

\clearpage

\end{document}