%%%%%%%%%%%%%%%%%%%%%%%%%%%%%%%%%%%%%%%%%%%%%%%%%%%%%%%%%%%%%%%%%%%%%%%%%%%%%
%% Original default rstudio/pandoc latex file
%% upated by @jhollist 09/15/2014
%% inspired by @cboetting https://github.com/cboettig/template and
%% @rmflight blog posts:
%% http://rmflight.github.io/posts/2014/07/analyses_as_packages.html 
%% http://rmflight.github.io/posts/2014/07/vignetteAnalysis.html).  
%%%%%%%%%%%%%%%%%%%%%%%%%%%%%%%%%%%%%%%%%%%%%%%%%%%%%%%%%%%%%%%%%%%%%%%%%%%%%

\documentclass[11pt,a4paper]{article}
\usepackage[T1]{fontenc}
\usepackage{lmodern}
\usepackage{amssymb,amsmath}
\usepackage{ifxetex,ifluatex}
\usepackage{fixltx2e} % provides \textsubscript
% use upquote if available, for straight quotes in verbatim environments
\IfFileExists{upquote.sty}{\usepackage{upquote}}{}
\ifnum 0\ifxetex 1\fi\ifluatex 1\fi=0 % if pdftex
  \usepackage[utf8]{inputenc}
\else % if luatex or xelatex
  \ifxetex
    \usepackage{mathspec}
    \usepackage{xltxtra,xunicode}
  \else
    \usepackage{fontspec}
  \fi
  \defaultfontfeatures{Mapping=tex-text,Scale=MatchLowercase}
  \newcommand{\euro}{€}
\fi
% use microtype if available
\IfFileExists{microtype.sty}{\usepackage{microtype}}{}
\usepackage[margin=1in]{geometry}
\usepackage{longtable,booktabs}
\usepackage{graphicx}
% Redefine \includegraphics so that, unless explicit options are
% given, the image width will not exceed the width of the page.
% Images get their normal width if they fit onto the page, but
% are scaled down if they would overflow the margins.
\makeatletter
\def\ScaleIfNeeded{%
  \ifdim\Gin@nat@width>\linewidth
    \linewidth
  \else
    \Gin@nat@width
  \fi
}
\makeatother
\let\Oldincludegraphics\includegraphics
{%
 \catcode`\@=11\relax%
 \gdef\includegraphics{\@ifnextchar[{\Oldincludegraphics}{\Oldincludegraphics[width=\ScaleIfNeeded]}}%
}%
\ifxetex
  \usepackage[setpagesize=false, % page size defined by xetex
              unicode=false, % unicode breaks when used with xetex
              xetex]{hyperref}
\else
  \usepackage[unicode=true]{hyperref}
\fi
\hypersetup{breaklinks=true,
            bookmarks=true,
            pdfauthor={},
            pdftitle={Enough is enough: How to sample plant-pollinator networks to make relative comparisons},
            colorlinks=true,
            citecolor=blue,
            urlcolor=blue,
            linkcolor=magenta,
            pdfborder={0 0 0}}
\urlstyle{same}  % don't use monospace font for urls
\setlength{\parindent}{0pt}
\setlength{\parskip}{6pt plus 2pt minus 1pt}
\setlength{\emergencystretch}{3em}  % prevent overfull lines
\setcounter{secnumdepth}{0}

%%%%%%%%%%%%%%%%%%%%%%%%%%%%%%%%%%%%%%%%%%%%%%%%%%%%%%%%
%Changes borrowed from @cboettig, added by @jhollist 
% A modified page layout 
\textwidth 6.75in
\oddsidemargin -0.15in
\evensidemargin -0.15in
\textheight 9in
\topmargin -0.5in
\usepackage{lineno} % add 
  \linenumbers % turns line numbering on 
%%%%%%%%%%%%%%%%%%%%%%%%%%%%%%%%%%%%%%%%%%%%%%%%%%%%%%%%

%%%%%%%%%%%%%%%%%%%%%%%%%%%%%%%%%%%%%%%%%%%%%%%%%%%%%%%%
%%Packages and layout changes by @jhollist 09/15/2014
\usepackage{ragged2e}
\usepackage[font=normalsize]{caption}
  \usepackage[doublespacing]{setspace}
\usepackage{parskip}
\usepackage{fancyhdr}
\pagestyle{fancy}
\fancyhf{}
\renewcommand{\headrulewidth}{0pt}
  \rfoot{\today}
\lfoot{\thepage}
%%Changed default abstract width and added lines
\renewenvironment{abstract}{
  \hfill\begin{minipage}{1\textwidth}
  \rule{\textwidth}{1pt}\vspace{5pt}
  \normalsize
  \begin{justify}
  \bfseries\abstractname\vspace{5pt}
  \end{justify}}
  {\par\noindent\rule{\textwidth}{1pt}\end{minipage}
}
%%%%%%%%%%%%%%%%%%%%%%%%%%%%%%%%%%%%%%%%%%%%%%%%%%%%%%%%

\title{Enough is enough: How to sample plant-pollinator networks to make
relative comparisons}
\author{
Ignasi Bartomeus
Cristina Botías
Miguel Ángel Collado
Óscar Godoy
Ainhoa Magrach
Curro Molina
Néstor Pérez-Méndez
Francisco Rodríguez-Sánchez
}
\date{}
% Allowing for landscape pages
\usepackage{lscape}
\newcommand{\blandscape}{\begin{landscape}}
\newcommand{\elandscape}{\end{landscape}}

% Left justification of the text: see https://www.sharelatex.com/learn/Text_alignment
% \usepackage[document]{ragged2e} % already in the latex template
\newcommand{\bleft}{\begin{flushleft}}
\newcommand{\eleft}{\end{flushleft}}

\begin{document}
%%Edited by @jhollist 09/15/2014
%%Adds title from YAML
\begin{singlespace}
\begin{center}
\huge Enough is enough: How to sample plant-pollinator networks to make
relative comparisons
\end{center}
%%Adds Author, correspond email asterisk, and affilnum from YAML
\begin{center}
\large
Ignasi Bartomeus \textsuperscript{*} \textsuperscript{1} 
Cristina Botías \textsuperscript{1} 
Miguel Ángel Collado \textsuperscript{1} 
Óscar Godoy \textsuperscript{2} 
Ainhoa Magrach \textsuperscript{1,3} 
Curro Molina \textsuperscript{1} 
Néstor Pérez-Méndez \textsuperscript{4} 
Francisco Rodríguez-Sánchez \textsuperscript{1} 
\end{center}
%%Adds affiliations from YAML
\begin{justify}
\footnotesize \emph{ 
\\*
\textsuperscript{1}Dept. Ecología Integrativa, Estación Biológica de Donana, Consejo
Superior de Investigaciones Científicas, Avda. Américo Vespucio 26,
E-41092 Sevilla, Spain\\*
\\*
\textsuperscript{2}Instituto de Recursos Naturales y Agrobiología, Consejo Superior de
Investigaciones Científicas, Avda. Reina Mercedes XX, XXXXX Sevilla,
Spain\\*
\\*
\textsuperscript{3}Euskadi\\*
\\*
\textsuperscript{4}Argentina\\*
}
%%Adds corresponding author email(s) from YAML
\newcounter{num}
\setcounter{num}{1}
\\[0.1cm]
\footnotesize \emph{ 
\ifnum\value{num}=1%
\textsuperscript{*} corresponding author:
\fi
\href{mailto:ibartomeus@ebd.csic.es}{\nolinkurl{ibartomeus@ebd.csic.es}}
\stepcounter{num}
}
\end{justify}
%%Adds date from YAML
\normalsize

\end{singlespace}


\singlespace

\vspace{2mm}

\hrule

Characterising complex networks on interacting species is challenging.
Two major stumbling blocks are a) correctly identifying species identity
and b) having a sufficient sampling size for detecting all occurring
links. A revisión of recently published plant-pollinator networks shows
that most published networks group several species into morphospecies or
even work at the genus level. In addition, when evaluated, sampling
completness is never reached. This has clear implications for describing
the network structure, but it is unknown how it affects relative
comparisions among networks. Ecologists are often more interested in the
relative comparision among communities with contrasting environments or
treatments rather than in the absolute values. When two networks are
sampled using the same methods, low taxonomic resolution and limited
sampling effort may not change its relative comparision, but this has
never been tested empirically. I use an intensive sampling of 16 fully
resolved plant-pollinator networks acorss an environmental gradient to
compare if the relative ranking of the main network propierties changes
when increasing taxonomic resolution (from morphospecies tos species)
and sampling effort. Then, I tested to which degree recently developed
techniques for predicting missing links enhance the relative coparisions
of poorly sampled networks. Determining which methods and indexes are
robust to relative comparisions is needed in order to empirically test
for pressing environmental changes that interacting communities are
facing.

\vspace{3mm}

\hrule

\emph{Keywords}: networks, pollination, mutualisms, sampling effort

\doublespace

\bleft

\section{INTRODUCTION}\label{introduction}

\section{METHODS}\label{methods}

\section{RESULTS}\label{results}

\section{DISCUSSION}\label{discussion}

Discuss.

\section{CONCLUSIONS}\label{conclusions}

\section{ACKNOWLEDGEMENTS}\label{acknowledgements}

\section{REFERENCES}\label{references}

\hypertarget{refs}{}

\eleft

\clearpage

\listoftables

\newpage

\begin{longtable}[]{@{}rrrrl@{}}
\caption{A glimpse of the famous \emph{Iris} dataset.}\tabularnewline
\toprule
Sepal.Length & Sepal.Width & Petal.Length & Petal.Width &
Species\tabularnewline
\midrule
\endfirsthead
\toprule
Sepal.Length & Sepal.Width & Petal.Length & Petal.Width &
Species\tabularnewline
\midrule
\endhead
5.1 & 3.5 & 1.4 & 0.2 & setosa\tabularnewline
4.9 & 3.0 & 1.4 & 0.2 & setosa\tabularnewline
4.7 & 3.2 & 1.3 & 0.2 & setosa\tabularnewline
4.6 & 3.1 & 1.5 & 0.2 & setosa\tabularnewline
5.0 & 3.6 & 1.4 & 0.2 & setosa\tabularnewline
5.4 & 3.9 & 1.7 & 0.4 & setosa\tabularnewline
\bottomrule
\end{longtable}

\newpage

\begin{longtable}[]{@{}lrrrrrrrrrrr@{}}
\caption{Now a subset of mtcars dataset.}\tabularnewline
\toprule
& mpg & cyl & disp & hp & drat & wt & qsec & vs & am & gear &
carb\tabularnewline
\midrule
\endfirsthead
\toprule
& mpg & cyl & disp & hp & drat & wt & qsec & vs & am & gear &
carb\tabularnewline
\midrule
\endhead
Merc 280 & 19.2 & 6 & 167.6 & 123 & 3.92 & 3.440 & 18.30 & 1 & 0 & 4 &
4\tabularnewline
Merc 280C & 17.8 & 6 & 167.6 & 123 & 3.92 & 3.440 & 18.90 & 1 & 0 & 4 &
4\tabularnewline
Merc 450SE & 16.4 & 8 & 275.8 & 180 & 3.07 & 4.070 & 17.40 & 0 & 0 & 3 &
3\tabularnewline
Merc 450SL & 17.3 & 8 & 275.8 & 180 & 3.07 & 3.730 & 17.60 & 0 & 0 & 3 &
3\tabularnewline
Merc 450SLC & 15.2 & 8 & 275.8 & 180 & 3.07 & 3.780 & 18.00 & 0 & 0 & 3
& 3\tabularnewline
Cadillac Fleetwood & 10.4 & 8 & 472.0 & 205 & 2.93 & 5.250 & 17.98 & 0 &
0 & 3 & 4\tabularnewline
Lincoln Continental & 10.4 & 8 & 460.0 & 215 & 3.00 & 5.424 & 17.82 & 0
& 0 & 3 & 4\tabularnewline
\bottomrule
\end{longtable}

\clearpage

\listoffigures

\newpage

\begin{figure}[htbp]
\centering
\includegraphics{output/figures/Fig1-1.pdf}
\caption{Just my first figure with a very fantastic caption.}
\end{figure}

\newpage

\blandscape

\begin{figure}[htbp]
\centering
\includegraphics{output/figures/Fig2-1.pdf}
\caption{Second figure in landscape format.}
\end{figure}

\elandscape

\clearpage

\end{document}